\documentclass{ltjsarticle}
\usepackage{amsmath}

\begin{document}
% section0〜section1-2
はろー\\
ほげほ
はろー
\\ \\
\noindent
$3+5-2=6$\\
$6\times 2\div 3 = 4$
\\ \\
\noindent
$\frac{1}{2}+\frac{1}{3}+\frac{1}{6}=1$\\
$1+\dfrac{1}{2}+\dfrac{1}{3}+...=-\dfrac{1}{12}$\\
$1+\cfrac{1}{2}+\dfrac{1}{3}+...=-\cfrac{1}{12}$
\\ \\
\noindent
(frac) 無限級数の和の例として$1+\frac{1}{2}+\frac{1}{4}+\frac{1}{8}+...=2$が成り立つ。\\ \\
(dfrac) 無限級数の和の例として$1+\dfrac{1}{2}+\dfrac{1}{4}+\dfrac{1}{8}+...=2$が成り立つ。
\\ \\
\noindent
$\dfrac{36}{11}$を正則連分数展開すると$\dfrac{36}{11}=3+\cfrac{1}{3+\dfrac{1}{1+\frac{1}{2}}}$となる。
\end{document}