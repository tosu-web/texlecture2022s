\documentclass{ltjsarticle}
\usepackage{amsmath}

\begin{document}
%実践1
%問題1-1
\noindent
$f(x)=\dfrac{2}{x}$とすると、
$\dfrac{f(2)}{f(4)}=\dfrac{1}{\frac{1}{2}}=2$であり、
$\dfrac{f(2)}{f(8)+f(8)}=\dfrac{1}{\frac{1}{4}+\frac{1}{4}}=2$でもある。
\\ \\
%問題1-2
\noindent
$S(k)=\dfrac{1}{1\times 2}+\dfrac{1}{2\times 3}+...+\dfrac{1}{k(k+1)}$
とすると、$S(k)=1-\dfrac{1}{2}+\dfrac{1}{2}-...-\dfrac{1}{k+1}=1-\dfrac{1}{k+1}$なので
数列$S(k)$の値は1に近づく。
\\ \\
%問題1-3(問題文)
$\dfrac{1}{(x-2)x(x+2)}=\dfrac{a}{x-2}+\dfrac{b}{x}+\dfrac{c}{x+2}$が$x$についての恒等式となるような定数$a,b,c$の値を求めよ。\\
\hfill (19' 工学院大学・文章改変)
\noindent
\end{document}