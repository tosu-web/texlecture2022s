\documentclass{ltjsarticle}
\usepackage{amsmath}
\usepackage{amssymb}

\begin{document}
%section2
\noindent
原点中心の単位円と直線$y=-2x$の交点は
$(\pm\dfrac{1}{2},\mp\dfrac{\sqrt{3}}{2})$ (複号同順)である。
\\ \\
\noindent
$\sqrt{\dfrac{a}{b}}=\dfrac{\sqrt{a}}{\sqrt{b}}$\\
$\sqrt[3]{5\sqrt{2}-7}=\sqrt{2}-1$
\\ \\
\noindent
不等式$x\geq\dfrac{1}{x}$の解は$-1\leq x<0,1\leq x$である。\\
$a\leqq b\leqq c$ならば$c\geqq a$である。
\\ \\
\noindent
$a_n=2^n$による等比数列において、$a_{11}=2^{11}=2048$である。\\
$\sqrt{2}^{\sqrt{2}}$は無理数か。
\\ \\
%実践2
%問題2-1
\noindent
2次方程式$x^2-3x+1=0$の解は$x=\dfrac{3\pm\sqrt{5}}{2}$である。
\\ \\
%問題2-2
\noindent
$\alpha =\sqrt[3]{5\sqrt{2}+7}-\sqrt[3]{5\sqrt{2}-7}$のとき、
$\alpha =2$である。
\\ \\
%問題2-3
\noindent
$0<|x-x_0|<\delta$ならば$|f(x)-L|<\varepsilon$が成り立つ。
\\ \\
%問題2-4
\noindent
3次方程式$x^3+3x^2-x+3\geq 0$を解け。
\\ \\
\noindent
$\tan\theta = \dfrac{\sin\theta}{\cos\theta}$\\
$\sin 2\theta =2\sin\theta\cos\theta$
\\ \\
\noindent
$\sin (90^{\circ}-\theta )=\cos\theta$
\\ \\
\noindent
$\log MN=\log M+\log N$\\
$\log _e x=\ln x$
\\ \\
%実践3
%問題3-1
\noindent
$2\leqq \log _{10}N<3 \Leftrightarrow 10^2\leqq N<3$
\\ \\
%問題3-2
\noindent
$\tan ^2\dfrac{\alpha}{2}=\dfrac{1-\cos\alpha}{1+\cos\alpha}$
\\ \\
%問題3-3
\noindent
不等式$\log _2\dfrac{x-6}{x-4}+\dfrac{\log _{x-4}x}{\log _{x-4}2}<2$を解け。\\
\hfill (19' 明治大・文系)
\end{document}