\documentclass{ltjsarticle}
\usepackage{color}
\usepackage{amsmath}
\usepackage{amssymb}
\title{シケタイのためのTeXゼミ}
\author{ますれー}
\date{2022年9月1日}

\begin{document}
%section3
\maketitle
吾輩は猫である。...
\\ \\
\tableofcontents
\noindent
\part{ぱーと}
% \chapter{ちゃぷたー}
\section{せくしょん}
\subsection{さぶせくしょん}
ほげほげほげほげほげほげほげほげ
\\ \\
\noindent
\section*{はじめに}
ほげほげと言えば何でも解決できる
\clearpage

\noindent
{\small もじサイズ}\\
もじサイズ\\
{\large もじサイズ}\\
{\Large もじサイズ}\\
{\LARGE もじサイズ}
\\ \\
\noindent
{\it イタリック italic}\\
{\bf ボールド}
\\ \\
% \usepackage{color}をする
\textcolor{red}{赤}い\textcolor{red}{りんご}と
\textcolor{blue}{青}い\textcolor{green}{りんご}
\\ \\
\noindent
ここは左揃え
\begin{center}
ここは中央揃え
\end{center}
\hfill ここは右揃え
\\ \\
\noindent
\begin{center}
(1つ目の方法) $e^{ix}=\cos x+i\sin x$
\end{center}
(2つ目の方法) $$e^{ix}=\cos x+i\sin x$$
\\ \\
\noindent
同じfracでも文中は$\frac{a}{b}$で独立形式は$$\frac{a}{b}$$
\\ \\
\noindent
fracの文中形式$\frac{a}{b}$でもうまくやれば$\displaystyle\frac{a}{b}$
\\ \\
\noindent
\begin{table}[htb]
\caption{1つ目の図}
\centering
\begin{tabular}{lll}
    あ & い & う \\
    え & お & か
\end{tabular}
\caption{下にもつけられる}
\end{table}
\\ \\
\noindent 
\begin{table}[htb]
\caption{2つ目の図}
\centering
\begin{tabular}{|l||l|l|} \hline
    あ & い & う \\ \hline
    え & お & か \\ \hline
\end{tabular}
\end{table}
\\ \\
\noindent
$$iphone の camera の数多すぎ$$
$$\mathrm{iphone} の camera の数多すぎ$$
\\ \\
\noindent
$$\left\{ \left( \dfrac{1}{2}+x\right) +y \right\}$$
\end{document}